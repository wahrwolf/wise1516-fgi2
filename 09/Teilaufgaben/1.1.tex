\begin{align*}
B=\{
&\begin{pmatrix}
0\\0\\3\\0\\2
\end{pmatrix},
\begin{pmatrix}
0\\0\\3\\1\\1
\end{pmatrix},
\begin{pmatrix}
0\\0\\3\\2\\0
\end{pmatrix},\\
&\begin{pmatrix}
0\\5\\2\\0\\2
\end{pmatrix},
\begin{pmatrix}
0\\5\\2\\1\\1
\end{pmatrix},
\begin{pmatrix}
0\\5\\2\\2\\0
\end{pmatrix},\\
&\begin{pmatrix}
0\\10\\0\\0\\0
\end{pmatrix},
\begin{pmatrix}
8\\0\\0\\0\\0
\end{pmatrix},
\begin{pmatrix}
2\\7\\0\\0\\0
\end{pmatrix},
\begin{pmatrix}
6\\2\\0\\0\\0
\end{pmatrix}
\}
\end{align*}
Die Ersten Beiden Zeilen Beschreiben die Markierungern, bei denen c für unbeschränktheit in $p_4$ sorgt.\\
Die Letzte Zeile Beschreibt die möglichkeiten, bei denen der Zyklus a,b für unbeschränktheit z.b. in $p_3$ sorgt.