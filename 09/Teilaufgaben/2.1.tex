Um Die Reihenfolge der einträge eindeutig zu machen mit zeilen bzw. spalten beschriftung
\[
\begin{pmatrix}
&t_1&t_2&t_3&t_4&t_5&t_6\\
pa&-1&0&1&-1&0&1  \\
p_1&+1&-1&0&0&0&0  \\
p_2&0&1&-1&0&0&0  \\
p_3&0&0&0&1&-1&0  \\
p_4&0&0&0&0&1&-1  \\
pp&0&-1&1&0&-4&4
\end{pmatrix}
\]
Lösen des Gleichnugssystems
%WOLFRAMALPHA eingabe:
%{{1,0,1,-1,0,1 },{ +1,-1,0,0,0,0  },{ 0,1,-1,0,0,0  },{ 0,0,0,1,-1,0  },{  0,0,0,0,1,-1  },{  0,-1,1,0,-4,4 }}*{{x_1}, {x_2}, {x_3}, {x_4},{x_5},{x_6}}={{0},{0},{0},{0},{0},{0}}

\[
\begin{pmatrix}
-1&0&1&-1&0&1  \\
+1&-1&0&0&0&0  \\
0&1&-1&0&0&0  \\
0&0&0&1&-1&0  \\
0&0&0&0&1&-1  \\
0&-1&1&0&-4&4
\end{pmatrix}
*i=\begin{pmatrix}
0\\0\\0\\0\\0\\0
\end{pmatrix}
\]
Ergibt als Lösung für die S-Invarianten-Vektoren 
\[\begin{pmatrix}
0\\0\\0\\x\\x\\x
\end{pmatrix}mit\ x\in \mathbb{N}\]
Für Die T-Invarianten gilt:
\[
p*
\begin{pmatrix}
-1&0&1&-1&0&1  \\
+1&-1&0&0&0&0  \\
0&1&-1&0&0&0  \\
0&0&0&1&-1&0  \\
0&0&0&0&1&-1  \\
0&-1&1&0&-4&4
\end{pmatrix}
=\begin{pmatrix}
0\\0\\0\\0\\0\\0
\end{pmatrix}^T
\]
\[
\begin{pmatrix}
0&0&0&x&4x&x
\end{pmatrix} mit\ x\in \mathbb{N}\]