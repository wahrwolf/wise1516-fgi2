\begin{align*}
sat(Error)&=\{c_4\}\\
sat(\neg Battery)&=\{c_5\}\\
sat(On)&=\{c_1,c_2,c_3\}\\
\end{align*}


Implikation als "{}wenn , dann"{}:\\
\\
Die Formel Bedeutet. Immer, wenn Error gilt, dann gilt, falls im nächsten schritt Nicht Battery gilt, irgentwann On.\\
\\
Oder Mit anderen Worten:\\
\\
Immer wenn es einen Fehler gab und im nächsten Schritt die batterie Entfernt wurde, so ist das Handy irgentwann wieder Eingeschaltet.\\
\\
In Einer unendlichen Folge ist diese Formel immer gültig:
$G(Error \Rightarrow ((X\neg Battery)\Rightarrow FOn))$
Falls Error Falsch ist, so gilt die Formel\footnote{nach definition von $\Rightarrow$}, daher schauen wir uns nur den Teil einer Rechnung an, in dem Error Wahr ist. Das Ist nur in $c_4$ der Fall. nach $c_4$ kann in der Rechnung $c_4$ oder $c_5$ folgen.\\
Falls $c_4$ folgt, so ist $(X\neg Battery)$ nicht erfüllt und diese Implikation also Wahr.\\
Falls $c_5$ folgt, so ist $(X\neg Battery)$ wahr, daher ist zu überprüfen, ob auch $FOn$ gilt. Dies ist der Fall, da die einzige möglichkeit wie die Rechnung fortgesetzt werden kann $c_0c_1$ ist und in $c_1$ On gilt