\subsubsection*{(a)}

\[\mathcal{B}_1= \{ (q_1,p_1),(q_2,p_2),(q_3,p_3),(q_4,p_4),(q_5,p_5),(q_3,p_4 ),(q_4,p_2)\} \]

\[\mathcal{B}_2= \{ (p_1,q_1),(p_2,q_2),(p_3,q_3),(p_4,q_4),(p_5,q_5),(p_3,q_4 ),(p_4,q_2)\} \]

\subsubsection*{(b)}
$\mathcal{B}_1\cup \mathcal{B}_2$ ist lediglich die symetrische Hülle von $\mathcal{B}_1$. Das die Beziehung der Bisimilarität Symetrisch ist, wurde bereits in Aufgabe 1 benutzt.

\subsubsection*{(c)}
In $TS_3$ ist dann keine aktionsfolge xyyyyyyyyyyyyyx mehr möglich in $TS_1$ alelrdings schon. Daher kann keine Bisimulationsrelation gefunden werden, die Bedingung b) erfüllt.